% chapters/07_conclusions_recommendations.tex
% This corresponds to the "conclusions and recommendations" chapter you requested.

\chapter{Conclusions and Recommendations}
\label{chap:conclusions}

This chapter presents the main conclusions derived from the study and outlines practical recommendations for addressing the identified public health and environmental risks associated with the local dumpsite. The conclusions synthesize the key findings, while the recommendations are targeted toward policymakers, municipal authorities, public health institutions, and the community.

\section{Conclusions}
The findings of this study clearly demonstrate that the solid waste disposal site poses significant risks to both the environment and the health of surrounding communities. Several major conclusions can be drawn:

\begin{enumerate}
    \item The current operational practices at the dumpsite—characterized by uncontrolled open dumping, frequent burning of waste, and an absence of engineering controls—create multiple pathways for human exposure to contaminants. These include air pollution from smoke and dust, proliferation of disease vectors such as flies and rodents, and potential contamination of surface and groundwater by leachate.

    \item Adverse health impacts are strongly reflected in the community survey data. Households residing in closer proximity to the dumpsite reported substantially higher rates of respiratory illnesses and gastrointestinal conditions. This statistically significant association indicates that inadequate solid waste management is a direct contributor to heightened health vulnerability in the community.

    \item While community awareness of the health risks associated with improper waste disposal is relatively high, this knowledge does not translate into safer household practices like waste segregation. This gap is primarily due to the lack of adequate municipal waste collection services and supporting infrastructure.

    \item Systemic institutional limitations, including insufficient municipal capacity, budgetary constraints, and weak enforcement of existing solid waste management policies, are root causes that perpetuate the problem.
\end{enumerate}

\section{Recommendations}
To mitigate the adverse health and environmental impacts, the following recommendations are proposed, addressing short-term, medium-term, and long-term actions:

\begin{itemize}
    \item \textbf{Implement Immediate Risk Reduction Measures:} Municipal authorities should immediately prohibit the open burning of waste. Access to the dumpsite should be controlled to limit unsafe scavenging activities, especially by vulnerable populations.

    \item \textbf{Improve Waste Collection and Introduce Segregation:} Expanding the coverage, frequency, and reliability of municipal waste collection services is critical. In parallel, the municipality should launch a pilot program for source segregation to reduce the volume of waste sent to the dumpsite.

    \item \textbf{Transition Towards Engineered Disposal Systems:} A phased, long-term plan should be developed to upgrade the site to a controlled or sanitary landfill. Initial steps should include the application of daily soil cover, followed by future implementation of leachate collection systems.

    \item \textbf{Protect and Integrate Informal Waste Workers:} Informal waste pickers should be formally recognized and integrated into the waste management system. Providing training on health and safety and ensuring access to personal protective equipment (PPE) is essential.

    \item \textbf{Strengthen Public Health Interventions:} Local health authorities should intensify disease surveillance in neighborhoods near the dumpsite. Community-based health education campaigns should be expanded to promote hygienic practices and raise awareness about exposure pathways.
\end{itemize}

