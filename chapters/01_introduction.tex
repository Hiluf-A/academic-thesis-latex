% chapters/01_introduction.tex

\chapter{Introduction}
\label{chap:introduction}

\section{Background of the Study}
\label{sec:background}
Solid waste management (SWM) remains one of the most pressing environmental and public health challenges in the 21st century, particularly in low- and middle-income countries experiencing rapid urbanization, population growth, and industrial expansion \cite{Kaza2018, Wilson2015}. Inadequate waste collection services, limited infrastructure, and poor enforcement of waste disposal regulations often result in practices such as open dumping, indiscriminate littering, and uncontrolled burning of solid waste. These practices contribute not only to environmental degradation—through air, water, and soil contamination—but also pose significant health risks to nearby populations due to the proliferation of vectors, exposure to hazardous substances, and respiratory problems caused by smoke and pollutants \cite{UNEP2015, Ferronato2019}.

In the African context, urban waste generation is growing at an alarming rate, outstripping the institutional and financial capacities of local governments to manage it effectively \cite{Hoornweg2012}. Ethiopia exemplifies this struggle, where urban solid waste management systems are often underfunded, poorly planned, and lack community participation. A study by Alemayehu et al. \cite{Alemayehu2021} highlighted that in many Ethiopian cities, less than 50\% of the generated waste is properly collected and managed, with the remainder either openly dumped or burned. The resulting health consequences include increased prevalence of respiratory illnesses, vector-borne diseases such as malaria and dengue, and gastrointestinal infections due to water contamination.

This study focuses on a representative peri-urban town in northern Ethiopia, which typifies this growing urban challenge. The town’s main waste disposal site operates as an open dump, lacking any engineered containment or environmental safeguards. Situated near residential areas, the site poses direct and indirect health risks to local communities, including children, the elderly, and waste pickers. Local observations and anecdotal evidence suggest a rise in respiratory issues, skin infections, and gastrointestinal diseases among the nearby population. However, there has been limited systematic research to assess the scale and nature of the public health impact resulting from this unmanaged waste site. This study aims to fill that gap.

\section{Statement of the Problem}
Solid waste management (SWM) in Ethiopia faces critical challenges due to rapid urbanization and limited infrastructure. Most urban areas rely on open dumping and burning, which leads to environmental pollution and increased public health risks. Poorly managed waste attracts disease vectors, emits toxic gases, and produces leachate that contaminates water sources, resulting in illnesses such as respiratory infections, waterborne diseases, and skin conditions \cite{MoUDC2017, WHO2018}.

While larger cities have received some attention, smaller towns remain under-researched. The study area exemplifies this issue. Its main disposal site is an open, unregulated dump that receives mixed, unsorted waste. The site lacks basic containment or safety measures, and open burning is common. Residents living nearby are exposed to air and water pollution, with informal reports suggesting a high prevalence of related illnesses. Despite clear health concerns, no systematic study had assessed the specific public health impacts. This lack of data limits the ability of local authorities to respond effectively.

\section{Objectives}
\subsection{General Objective}
The main objective of the study was to assess the impact of solid waste disposal practices at the local dumpsite on the public health of nearby communities, with the aim of generating evidence to support targeted environmental and health interventions.

\subsection{Specific Objectives}
\begin{enumerate}
    \item To examine the existing solid waste management system in the town, with an emphasis on disposal practices at the site.
    \item To identify and quantify the prevalence of key health conditions—such as respiratory, gastrointestinal, and dermatological illnesses—among residents living near the site.
    \item To assess community perceptions regarding the health impacts of living near the waste disposal site.
\end{enumerate}

\section{Research Questions}
\begin{enumerate}[label=\arabic*.]
    \item What were the solid waste disposal practices at the site, and how did they align with national and international public health standards?
    \item What were the most prevalent health issues reported by residents living in proximity to the waste site?
    \item Was there a statistically significant relationship between household proximity to the site and the incidence of specific health conditions?
\end{enumerate}

\section{Hypotheses}
\begin{enumerate}[label=H\arabic*.]
    \item The solid waste disposal practices at the site were environmentally inadequate and posed health risks to surrounding communities.
    \item Residents living closer to the site experienced a higher prevalence of respiratory, gastrointestinal, and dermatological conditions compared to those living farther away.
    \item The majority of community members living near the site perceived a direct connection between the waste disposal site and their family’s health issues.
\end{enumerate}

\section{Significance of the Study}
This study generates critical evidence to inform urban planning and public health policy. By assessing the health impacts of solid waste disposal practices, the research supports the development of more sustainable waste management systems and interventions aimed at reducing community exposure to environmental health risks. For public health authorities, the findings can guide the design of effective health education campaigns, strengthen disease surveillance, and enhance risk communication. Academically, the study contributes to the limited literature on environmental health in secondary towns of developing countries.

\section{Expected Output}
The research produced a comprehensive assessment of the public health impacts associated with the solid waste disposal practices at the local site. By generating both quantitative and qualitative data, the study identified operational deficiencies in waste management and quantified the prevalence of key health conditions. The analysis determined whether a statistically significant relationship existed between proximity to the site and adverse health outcomes. Based on these findings, the study provides actionable, context-specific policy recommendations to improve waste management and reduce public health risks.
