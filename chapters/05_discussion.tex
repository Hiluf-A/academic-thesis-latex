% chapters/05_discussion.tex
% This chapter provides an in-depth discussion of the research findings.

\chapter{Discussion}
\label{chap:discussion}

This chapter interprets the findings presented in Chapter 4, contextualizing them within existing academic literature and theoretical frameworks. It also explores the public health and policy implications of the results.

\section{Comparison with Prior Studies}
The conditions at the study site exemplify the challenges typical of open dumping in developing countries. The high prevalence of self-reported respiratory infections (63.3\%) is consistent with studies in other parts of Africa that have linked proximity to dumpsites with increased rates of respiratory and gastrointestinal diseases \cite{Ndejjo2016}. The observed practice of open waste burning is a known source of particulate matter and toxic gases, corroborating global evidence on the health risks of uncontrolled combustion \cite{Porta2009}.

Similarly, the high prevalence of diarrheal diseases (53.3\%) aligns with literature that links groundwater contamination by leachate to waterborne illnesses \cite{Ferronato2019}. The observed presence of vectors like flies and rodents reinforces findings from studies at other major Ethiopian dumpsites, such as Koshe and Hawassa, where vector proliferation was also identified as a significant health hazard \cite{Bogale2019, Abebe2018}.

The strong perception (85.3\%) among residents that the dumpsite negatively affects their health mirrors the Environmental Health Paradigm, which posits that communities often directly recognize environmental stressors as key determinants of their well-being. This validates the Exposure Pathway Model in this context: inhalation (of smoke and dust), ingestion (of contaminated water or food), dermal contact, and vector bites all emerged as plausible routes of exposure according to both observations and community reports.

\section{Implications for Public Health and SWM}
The findings highlight several urgent health and policy challenges:
\begin{itemize}
    \item \textbf{Public Health Burden:} The dominance of Acute Respiratory Infections (ARIs) and diarrheal diseases suggests ongoing community exposure to air and water contamination originating from the dumpsite. It is likely that seasonal variations exacerbate these risks, with increased dust and burning during the dry season and wider leachate spread during rainy months.
    
    \item \textbf{Community Perceptions vs. Practice:} The strong awareness of health risks contrasts sharply with the low rate of household waste segregation. This indicates a significant gap between knowledge and practice, likely driven by the lack of reliable municipal collection infrastructure and incentives for segregation.
    
    \item \textbf{Institutional Gaps:} The absence of basic engineering controls (such as liners, leachate management systems, or gas vents) at the site magnifies the environmental and health risks. Furthermore, the practice of informal scavenging, while providing a livelihood for some, perpetuates cycles of exposure and contributes to the dispersal of waste into the surrounding environment.
    
    \item \textbf{Financing Opportunities:} The finding that nearly half of the respondents expressed a willingness to pay for better services suggests a potential avenue for developing sustainable funding mechanisms for SWM, provided such systems are implemented with transparency and accountability.
\end{itemize}

Overall, the results reinforce the global consensus on the dangers of open dumping while illustrating the acute local realities in the study area. The evidence strongly supports the need for integrated waste management reforms, enhanced public health education, and institutional strengthening.

